% !TEX program = xelatex

% https://github.com/MCG-NKU/NSFC-LaTex
% by Ming-Ming Cheng https://mmcheng.net
% 关于VsCode LaTeX的配置 https://www.cnblogs.com/ourweiguan/p/11785660.html
% Windows下Tex Live最新版可以用pdfLatex快速编译和标准模板相似度更高的文档。



\documentclass[12pt]{article}


\usepackage{nsfc}


%\usepackage{fontspec}
%\usepackage{xcolor}
%\defaultfontfeatures{Ligatures=TeX}




\newcommand{\addImg}[2][1.0]{\includegraphics[width=#1\linewidth]{#2}}
\newcommand{\cmm}[1]{\textcolor[rgb]{0,0.6,0}{CMM: #1}}
\newcommand{\todo}[1]{{\textcolor{red}{\bf [#1]}}}
\newcommand{\myPara}[1]{\paragraph{#1:}}
\newcommand{\myEmph}[1]{\textbf{\textcolor[rgb]{0,0,0.25}{#1}}}
\newcommand{\mySec}[1]{\vspace{0.15in} \noindent \myEmph{$\square$~#1}}

\graphicspath{{figures/}}


\begin{document}




%\if 0
\begin{center}\bf\Large ***识别研究 (提交时注释掉) \\

    具间接信号产生的生物趋化模型解的爆破研究

Research on target recognition *** 
\end{center}





\textbf{摘要}:
%
“用……方法(手段)进行……研究,探索/证明……问题,
对阐明……机制/揭示……规律有重要意义,为……奠定基础/提供……思路 ”。
作用:画龙点睛。
效果:引发评议专家兴趣,使其产生探个究竟的好奇心——“他到底要怎么做?”
摘要字少,但切忌平淡无奇(要勾起评委浓厚兴趣)。
\myEmph{一定要语气坚定,旗帜鲜明,字数有限,资源宝贵,要特别注意重点突出,
讲明现状、意义、课题主要研究目标、内容、思路和预期结果}。



\textbf{Abstract}:
***


科学问题属性:聚焦前沿,独辟蹊径



计算机视觉研究正在经历一个各种检测识别技术大规模进入实际应用的时期。
然而,如何面对绝大多数实际应用中样本数据不足、场景类别不确定、
采集数据质量低等现实问题,
研究不确定环境下小样本目标识别,
依然是计算机视觉领域的世界级科技前沿问题。
%
该问题的解决,有望带领计算机视觉技术进入更为广阔的深水区。
申请人团队拟基于图像场景理解领域内世界前沿的学术成果积累,
在时效性约束和资源受限条件下,研究不确定性的建模及其诱导的
鲁棒深度学习技术,开发适应能力强的神经网络技术,
并利用通用经验知识,推动不确定环境下小样本目标识别理论和方法的突破。
%
相对于现有工作,本项目拟研究的技术具有鲜明的引领性。
不同于传统技术尽量避免不确定性,
该研究计划系统地对不确定性进行建模,并且利用不确定性的特点开发鲁棒的智能算法,
因而,该项目独辟蹊径,有望引领该领域的国际前沿研究发展。


\clearpage
%\fi


%%%%%%%%% TITLE
\title{报告正文}

\maketitle

%\emph{\large 参照以下提纲撰写,要求内容翔实、清晰,层次分明,标题突出。
%\nsfClr{请勿删除或改动下述提纲标题及括号中的文字。}}


\ContentDes{(一)立项依据与研究内容(建议$8000$字以内):}


\NsfcSection{1}{项目的立项依据}{
(研究意义、国内外研究现状及发展动态分析,需结合科学研究发展趋势来论述科学意义;
或结合国民经济和社会发展中迫切需要解决的关键科技问题来论述其应用前景。
附主要参考文献目录);}



题目是你对评审专家说的第一句话。需要创新、创新、再创新!!
尽量回答“干什么、对象是什么、用什么方法、解决什么问题”。
简洁明确,具体清楚。不宜过长,不宜出现过多的关键词,但最好要有新意的关键词出现,
包含研究视角、方法和研究对象的创新,
最好能让专家一看到“题目名称”就能基本了解本申请重点要研究的问题。 
忌讳项目名称重复,即使所提出的与以前资助项目研究内容有所不同,甚至有所创新,
但名称重复很难给人以新意。
\myEmph{建议检索类似课题历年资助情况,避免重复!!}
%
在选择研究题目(或方向)时,应本着“扬长避短”的原则,尽量结合自己的研究基础;
缺乏一定科学(研究)基础的“创新”是不成立的,许多情况甚至是“空想”。
选题最好以问题为导向,不要新以技术、新方法的应用为导向!
忌盲目追求“学科前沿”和“研究热点”问题。


写本子最难的是\myEmph{想题目}和\myEmph{写摘要},
需要广泛快速阅读近几年的文献。
可以考虑找到一个具体的场景,用场景将本子的研究内容串起来。
也可以围绕一个关键科学问题,将多个方面的研究内容有机结合起来。
建议先画两个图:\myEmph{研究内容框架图}和\myEmph{详细技术路线图}。
有了这两个总蓝图,整个本子写起来会更加的有条理,避免各种混乱和不一致。
%
为了方便专家快速抓住重点,建议把\myEmph{重点内容标粗},
让专家即使只阅读非常少的加粗字体,也可以获得判断本子优劣的足够关键信息。


我在参加基金委重点项目会评之前,
基金委领导特意给评委们科普了基金委最看重的\myEmph{灵魂六问},请专家们重点关注:
\begin{itemize}
    \item 该项目想做什么?请用大同行能够理解的术语表达您的研究目标。
    \item 目前的做法有哪些局限性?
    \item 您的方法有什么独特性,为什么您认为它会成功?
    \item 谁关心该项目取得的成果?
    \item 如果您成功了,对该领域有的推动作用是什么?
    \item 中期和期末,如何检查该项目计划成功与否?
\end{itemize}

在撰写基金的过程中,各部分需要重点阐明
\begin{itemize}
    \item 立项依据:为什么做
    \item 研究内容:做什么
    \item 研究目标:做到什么程度
    \item 研究方案:如何做
    \item 工作基础等:我能做该项目
\end{itemize}


具体到立项依据,重点阐述清楚:
\myEmph{为什么要做这个课题?重大需求、存在问题、有解决思路。谁会关心该项目取得的成果?}
让评审者读了申请书以后要有如下感觉:
这个研究很重要,国内外都在做,但有要害问题没有解决,
申请人提出了很好的解决途径,思路很独特且合理,
若沿着这条思路做几个方面的研究,有解决希望。


对基础研究,着重结合国际科学发展趋势,论述项目科学意义对应用基础研究,
着重结合学科前沿,围绕国民经济和社会发展中的重要科技问题论述其应用前景。
%
立项依据部分应该包括:
\begin{itemize}
    \item 立项意义;
    \item 国外同类研究状况;
    \item 国内同类研究状况;
    \item 本课题组的研究基础和选题的依据。    
\end{itemize}
立项依据论述要简明扼要,有理有据。
要用准确的学术语言,将问题论述清楚,一般要考虑如下问题:
1、什么人在研究?研究了些什么?核心科学问题是什么? 
2、人家怎么进行研究?解决了些什么问题?还有什么问题没解决? 
哪些问题是别人想到了的?但没有解决?
3、你考虑怎么解决?哪些问题是别人还没有想到的?你又是考虑怎么来解决?
4、如果您成功了,对该领域有什么推动作用?


\subsection{研究背景及科学意义}

\subsubsection{*题目*对**具有重要作用}

第一段:主要描述题目的主要工作,解释题目中涉及的术语,
然后说明输入和输出,说明目前题目的主要工作是如何实现的,
具有什么样的意义。

第二段:说明第一段描述的内容的局限性和不足,
引出本项目的主要工作。

第三段:说明解决上述局限性和不足,有什么实际应用效果。

第四段:说明该项目工作对***的重要作用。


\subsubsection{*题目*具有挑战性的问题}

项目需要那些技术,这些技术的复杂程度,面临的主要问题,
要使用一些术语来表述,最后要有结论性语言,例如:

\myEmph{因此,当前***技术很难应用于面向***的研究,
在***等方面也缺少针对 *** 的专门研究,
特别是缺乏针对***的有效方法。
这些局限性限制了***分析技术的发展 。
}


\subsubsection{***思想为***题目*研究提供了新途径}

一些最新的技术,有利于项目工作的解决。
最后务必有一段总结的语言,例如

\myEmph{本项目基于***数据,建立***,利用***表示和分析方法,
以***为核心,研究***演化规律,
在理论和技术层面都具有重要的引领和示范作用。}


\subsection{***题目*面临的挑战}

提出3-4个挑战,分别对应后面的主要研究内容,最后一定要有对挑战的总结描述,例如

\myEmph{因此,面向****的***,需要对***、***和***进行统一的概念描述与框架建模。
在此基础上,通过****是一个可行途径。}


\subsection{总结}

说明课题组具有解决上述挑战的能力,例如:

\myEmph{综上所述,***,本项目从***出发,面对***等挑战,
研究***关键科学问题。
项目拟在以下三方面开展研究:****}


项目组在***具有扎实的研究基础,承担了多项国家和省部级课题, 典型的有:***。
这些项目的完成,使项目组在***方面积累了丰富的研究经验,
也为该项目的实施奠定了良好的技术基础。

本项目直接面向***国家战略,可以为***等重要应用提供核心技术支撑,
对推进***的应用和发展具有重要的意义。

\subsection{国内外相关工作}

如果申请人从未在所申请项目的研究领域发表过一篇论文,
或者申请书中对国内外研究现状阐述不明,不附主要参考文献目录,
说明申请人在这一研究领域无研究工作基础,不具备实施该项目的研究能力。

在相关工作的评述中,应该尽量广泛的包含各方面的先进成果:
既有国际上顶级的研究,也得有国内最先进的成果。
\myEmph{建议同时包含自己的中文%\cite{21SC_WebSegE}
和英文%\cite{ChengZMHH10}
的代表作,
这些相关的代表作也标明自己在该领域的研究基础。}
可以引用几个代表性图,形象的展示一些重要的背景知识,方便大同行理解。

可以对相关工作进行分组介绍。
每组介绍之后简单总结一下现有工作和拟研究工作的关系。
\myEmph{现有工作存在那些不足需要进一步研究。
这些总结性的结论建议粗体强调,方便评审人迅速理解。}

{
\bibliographystyle{nsfc.bst}
\bibliography{Cmm}
}

\NsfcSection{2}{近期主要研究工作情况}{(着重阐述近5年来在基础研究方面所开展的研究工作,以及目前的研究进展、取得的成绩、研究成果及价值和科学意义等。3000字左右)}

申请人近五年来一直学习偏微分方程领域的相关基本理论与方法,围绕非线性抛物型偏微分方程中的关键科学问题,聚焦Keller-Segel型生物趋化模型解的定性性质,通过比较方法、权函数法、能量方法、半群理论以及多种泛函不等式等数学工具与理论,系统研究了三类具有明确生物背景的趋化模型解的爆破理论,揭示了趋化模型解新的爆破模式,分别解释了非线性指数、初始种群总量以及空间维数对决定趋化模型定性性质的关键作用并刻画了趋化模型常数稳态解的不稳定性。相关成果在诸如血管生成和阿尔兹海默症等生理病理以及病虫害防治与预防上具有重要的价值。申请人撰写并投稿了六篇论文,其中三篇论文已经发表并被SCI检索,相关成果发表在Mathematical Models and Methods in Applied Sciences、Nonlinear Analysis和Discrete and Continuous Dynamical Systems - Series S等期刊上。%申请人还另外撰写并完成了三篇论文,都在投稿中。

研究背景:趋化指生物(包括细菌、细胞及其他单细胞或多细胞生命体)对外界环境中的化学物质刺激所产生的趋向性反应。趋化性是生物的本能反应之一,在自然界中广泛存在,比如细菌向有较高养分浓度的区域游走,或远离有害物质的地方。趋化性不仅对生物觅食、躲避天敌、规避有害物质、求偶等有关键作用,具有重要的生理病理学研究价值,而且在病虫害防治、生物除污、生物医疗、石油开采及病虫害防治等生产生活领域中具有重要的应用价值。

研究内容:Keller-Segel型生物趋化模型主要描述描述生物种群对自身产生的趋化因子有趋化性运动的现象,展现了扩散与(趋化诱导)聚集的竞争机制,具有丰富的动力学行为,特别会发生爆破现象——聚集的极端形式——这吸引了大批数学学者研究趋化模型解的定性理论。申请人主要在(1)基于细胞本身占有一定体积而提出的密度依赖趋化模型(也称为具体积填充效应的趋化模型)、(2)基于细胞的有限扩散速度而提出的通量限制趋化模型以及(3)基于种群表型多样性而提出的具间接信号产生的趋化模型上取得了一些研究进展。

\textbf{主要研究进展与创新成果:}
申请人主要研究了上述三类趋化模型的 J\"ager-Luckhaus 型抛物-椭圆变体以及具间接信号的完全抛物趋化模型的径向对称古典解的爆破性质。对于前者主要用比较方法侦测爆破,对于后者则主要用能量方法构造爆破解。先总结一下目前的研究结果(其中前三个结果已经发表在期刊上,后三个结果都在投稿中)。

(i)研究了二维及更高维空间中的小球区域中超临界密度依赖趋化模型的临界质量现象:存在一个质量阈值使得当初始总量高于该阈值并且初值单调非增,则解都会有限时刻爆破;存在初始总量低于该阈值且单调非增的初值使得解整体存在。

我们发现了在超临界非线性敏感度条件下高维空间中也存在一个新的临界质量现象,突破了非线性敏感度函数的带来的困难对质量分布函数满足的一点退化标量抛物型偏微分方程构造了一族稳态下解,为初始质量在决定趋化模型解的定性性质上的关键作用提供了更多的理论支持。

(ii)研究了三维及更以上的小球区域中超临界通量限制趋化模型常数稳态解的不稳定性:大质量常数稳态解是不稳定的;小质量常数稳态解是局部渐近稳定的。该结果对临界通量限制趋化模型也成立。

我们突破了非线性梯度依赖敏感度函数带来的技术困难,刻画了大质量常数稳态解的不稳定程度:即只要大质量的径向对称初值比同质量的常数稳态解在原点附近更集中,解就会在有限时刻爆破。上述大质量常数稳态解的不稳定性在临界敏感度的趋化模型中也成立,即只要大质量的径向对称初值比同质量的常数稳态解在原点附近更集中,解就会在有限或无穷时刻爆破。该结果正面回答了临界通量限制趋化模型的古典解能否发生爆破的公开问题~\cite{Marras2023}。我们的结果有助于研究珊瑚虫等生物受精以及血管生成等趋化现象。

(iii)研究了单位圆盘上间接信号趋化模型的趋化坍缩现象:对大质量的径向对称初值,只要比常数稳态解在原点附近聚集更多质量,解就会在无穷时刻依Dirac型分布函数的形态于原点处聚集所有初始总量。

我们运用构造稳态下解族侦测爆破的思路突破了质量分布函数满足的时间非局部抛物问题带来的困难,揭示了间接信号趋化模型不具有经典趋化模型趋化坍缩的质量量子化现象~\cite{Nagai2000,Suzuki2013},反映了间接信号趋化模型具有新的爆破模式,为控制与预防山松甲壳虫的集中爆发提供了理论依据。

(iv)研究了具间接信号的拟线性完全抛物趋化模型解的整体存在性与无界性:在二维及以上的有界光滑区域中,当趋化灵敏度较弱时解总是整体存在的;在四维及以上的小球区域中,当趋化灵敏度主导扩散强度时,存在径向对称初值使得解无界。

我们运用最大正则性理论研究解的整体适定性,并克服了间接信号机制的爆破抑制效应,运用能量方法研究稳态解的能量下界并构造一族大负值能量初值从而依归谬法侦测爆破。结合已有结果,我们的结果分别解释了解整体存在或有限时刻奇性形成以及解一致有界或有限或无穷时刻爆破的最优指数问题,有助于理解阿尔兹海默症的趋化机制。

(v)研究了具间接信号的抛物--抛物--椭圆趋化模型解的有限时刻爆破:在五维及以上的小球区域中,对任给质量,都存在具有该质量的径向对称初值使得解有限时刻爆破。

我们突破了高度耦合的空间非局部型的信号产生项带来的困难,运用能量方法建立了能量与其耗散率之间的一个次线性泛函不等式,基于此证明了低能量初值迫使有限时刻爆破发生,并证明了低能量初值在合适的拓扑中具有一定的稠密性。

% (vi)研究了单位圆盘上具临界初始质量的 J\"ager-Luckhaus 模型解的一致有界性:对任意具有临界质量的径向对称初值,解都是整体存在且一致有界的,还收敛到唯一的常数稳态解。

% 我们突破了临界质量给模型解定性性质带来的困难,对由质量分布函数满足的标量抛物问题运用比较方法,构造了临界质量条件下的一族稳态上解与稳态下解,给出了稳态问题径向对称解唯一性的新证明。该结果给出了具临界初始质量的趋化模型不会发生爆破的一个例子,不同于已有的相关结果。

%趋化模型大质量常数稳态解的不稳定性是广为人知的,比如见早期Nanjundiah的分析工作~\cite{Nanjundiah1973}。但在对该不稳定性更为确切与精细的刻画方面,令人满意的结果比较少。(i,ii,iii)研究了三类趋化模型的J\"ager-Luckhaus型变体的径向对称古典解的爆破问题,将最初由Winkler~\cite{Winkler2019}发现的新的临界质量现象,推广到具有一般超临界敏感度的趋化模型以及间接信号趋化模型中。这些结果还反映了大质量常数稳态解的不稳定程度,特别的,(iii)的结果


\NsfcSection{3}{拟开展的研究工作(模板标题)}{
(阐述下一步拟开展的研究工作的研究背景、主要内容、研究目标、研究方案和技术路线、特色与创新之处、研究可行性、研究基础和条件保障等,重点阐述拟开展的课题的科学前沿性,或是围绕我省经济和社会可持续发展的重大需求方面内容。5000字左右)}

\NsfcSection{31}{研究背景}{}

为了理解种群表型多样性如何影响种群自发聚集现象\cite{Macfarlane2022},本课题拟研究一类具间接信号产生的生物趋化模型解的定性性质,着重研究聚集现象与爆破性质。其典型形式为
\begin{align}
    \begin{cases}
      \label{sys: ks isp pppf}
        u_t =  \nabla\cdot(D(u)\nabla u - S(u)\nabla v) + f(u,w),& x\in\Omega, t>0,\\
        v_t =  \Delta v - v + w,& x\in\Omega,	 t>0,\\
        \tau w_t  = \varepsilon\Delta w - w + u, & x\in\Omega, t > 0, 
    \end{cases}
\end{align}
其中$\tau,\varepsilon\in\{0,1\}$。该模型是经典Keller-Segel趋化模型\cite{Keller1970}的简单扩展,其中种群被划分为两种不同的表型$u$和$w$:前一种执行趋化性运动,后一种则生产趋化因子。作为一个具体的应用例子,该模型描述了山松甲壳虫的集群攻击行为\cite{Strohm2013},其中$u=u(x,t)$表示飞行甲壳虫的密度,$w=w(x,t)$表示筑巢甲壳虫的密度,$v=v(x,t)$表示甲壳虫费洛蒙的浓度,$D(u)$表示扩散强度,$S(u)$表示趋化灵敏度,$f(u,w)$表示甲壳虫的出生、死亡以及筑巢与飞行甲壳虫之间的转化机制。该模型在描述动物的感知与记忆因素对其种群迁移运动的影响中也有重要应用\cite{Shi2021}。更多相关生物背景可参看近期综述文献\cite{Winkler2025}。

1970年,Keller和Segel提出经典趋化模型($D(u)\equiv1$,$S(u)\equiv u$,$f(u,w)\equiv0$并且$\tau=\varepsilon=0$)
\begin{equation}
    \label{sys: ks1970}
    \begin{cases}
        u_t = \nabla \cdot(\nabla u - u\nabla v), \\ 
        v_t = \Delta v - v + u,
    \end{cases}
\end{equation} 
来描述细胞的聚集现象,其中$u$表示微生物的种群密度,$v$表示化学信号的浓度\cite{Keller1970}。微生物一方面会随机游走,还会沿着自身释放的化学信号浓度的梯度方向运动。该自组织系统展现了扩散与(趋化诱导)聚集的竞争机制,吸引了大批数学学者研究趋化模型解的定性理论。特别的,有限时刻爆破现象——聚集的极端形式——会在某些条件下发生:在一维有界区域中,扩散主导聚集,解都是整体存在且一致有界的\cite{Osaki2001}。在平面区域中,当初始质量超过$4\pi$且不为其整数倍时,存在具有该质量的初值使得解在有限或无穷时刻爆破\cite{Horstmann2001};当初始质量低于$4\pi$时,解总是整体存在且一致有界的\cite{Nagai1997};而且在圆盘区域中,当初始质量超过$8\pi$,存在具有该质量的径向对称初值使得解在有限时刻爆破\cite{Mizoguchi2014},但当初始质量低于$8\pi$时,径向对称解总是整体存在且一致有界的\cite{Nagai1997}。在更高维空间,扩散受趋化作用支配,在径向对称假设下任给初始质量都存在具有该质量的径向对称初值使得解有限时刻爆破\cite{Winkler2013}。关于该模型还有很多有意思的工作,涉及各个方面,例如趋化坍缩现象\cite{Nagai2000},一致有界条件下\cite{Feireisl2007}、二维次临界质量\cite{Wang2019}以及高维小性条件假设下\cite{Cao2015}的渐近行为,有限时刻爆破的能量准则\cite{Mizoguchi2020},爆破速率\cite{Mizoguchi2020a}以及爆破形态\cite{Winkler2020}。

当考虑体积填充效应\cite{Painter2002}($D(u) = (1+u)^{-p}$并且$S(u)=(1+u)^{q-1}u$),经典趋化模型~\eqref{sys: ks1970}就变成了密度依赖趋化模型
\begin{equation}
    \label{sys: ksvf}
    \begin{cases}
        u_t = \nabla \cdot(D(u)\nabla u - S(u)\nabla v), \\ 
        v_t = \Delta v - v + u.
    \end{cases}
\end{equation} 
许多学者研究了该模型爆破的临界指标问题。在超临界指标$p+q>2/n$下,趋化主导扩散,在$n$维空间的小球中存在径向对称初值使得解有限或无穷时刻爆破\cite{Winkler2010},其中$n\geq2$,而且如果进一步假设$p>1$或$q>1$,存在径向对称初值使得解在有限时刻爆破\cite{Winkler2010a};当次临界条件$p+q<2/n$时,扩散支配趋化聚集,解总是整体存在且一致有界的\cite{Tao2012};临界情形$p+q=2/n$时,小质量初值总是整体存在且一致有界的,而存在大质量径向对称初值使得解无界\cite{Winkler2022}。另一方面,当趋化灵敏度较弱$q\leq0$时,解总是整体存在的,所以即使趋化强度较弱但仍主导扩散时,存在径向对称解在无穷时刻爆破\cite{Winkler2019}。而且对模型~\eqref{sys: ksvf}的抛物-椭圆简化问题,已经证明了只要$p+q>2/n$并且$q>0$,总存在径向对称初值使得解有限时刻爆破\cite{Winkler2010b}。因此学者提出猜想$q=0$是模型~\eqref{sys: ksvf}整体适定与有限时刻奇性形成的第二临界指标,并放松了模型~\eqref{sys: ksvf}有限时刻爆破的指标范围\cite{Cao2025}。

经典趋化模型没有考虑种群表型异性机制,不能满足更真实和复杂的建模需求。文献在研究山松甲壳虫的聚集攻击行为时,提出了具间接信号产生的抛物-抛物-ODE系统($D(u)\equiv1$,$S(u)\equiv u$,$\tau=1$但$\varepsilon=0$)。当考虑种群的出生与死亡时,即$f(u,w)=\mu u(1-u)$,其中$\mu>0$是个常数,在三维有界区域中解总是整体存在且一致有界的;而对经典趋化-生长系统,需要假设$\mu$大于某个给定常数才能得到解是一致有界的,这反映了间接信号趋化模型的时间松弛机制具有有界支持效应。不考虑种群的出生与死亡时$f(u,w)\equiv0$,Tao和Winkler最早研究该模型的的J\"ager-Luckhaus变体,即模型中的第二个方程替换为$0 = \Delta v  - \int_\Omega w dx/|\Omega| + w$,并发现在圆盘区域中,虽然径向对称解总是整体存在的,依然存在一个无穷时刻爆破的临界质量现象:存在超过$8\pi$质量的径向对称初值使得解在无穷时刻爆破,但当初始质量低于$8\pi$时,径向对称解总是一致有界的。申请人与合作者发现该模型的J\"ager-Luckhaus变体会发生无穷时刻在一点集中所有初值质量的新现象。在平面有界光滑区域中,虽然解总是整体存在的,但依然存在一个无穷时刻爆破的临界质量现象:当初始质量超过$4\pi$且不为其整数倍时,存在具有该质量的初值使得解在无穷时刻爆破;当初始质量低于$4\pi$时,解总是整体存在且一致有界的。当考虑类型转化机制时,爆破的临界质量与种群初始总量之和有关。因为间接信号机制具有爆破抑制效应,所以在三维及以上是否会发生有限时刻爆破依然是公开问题。

对于具间接信号产生的完全抛物趋化模型~\eqref{sys: ks isp pppf},其中$\tau=\varepsilon=1$,当$D(u)=1$并且$S(u)=u$时,由于抛物光滑化效应,模型在三维及以下的解总是整体存在且一致有界的,但存在一个四维临界质量现象:当初始质量低于$64\pi^2$时,解总是整体存在且一致有界的;存在超过$64\pi^2$但不为其整数倍质量的初值使得解在有限或无穷时刻爆破。当考虑体积填充效应时,即$D(u)=(1+u)^{-p}$且$S(u)=(1+u)^{q-1}u$,当扩散主导趋化时$p+q<\max\{1+2/n,4/n\}$,解都是整体存在且一致有界的。Tao和Winkler最早研究该模型的J\"ager-Luckhaus变体并考虑了潜在的转化机制,并发现在三维及以上的小球区域中,当$p+q>4/n$并且$q>2/n$时,存在径向对称初值使得解在有限时刻爆破,并且当$q<2/n$时,解都是整体存在且一致有界的。申请人与合作者证明了在四维及以上的小球区域中,当$p+q>4/n$并且$q<2/n$时,具间接信号产生的完全抛物问题存在径向对称初值使得解无穷时刻爆破。为了探索有限时刻爆破机制,申请人与合作者研究了模型~\eqref{sys: ks isp pppf}的一个Nagai型变体$\tau=0$,并证明了在五维及以上的小球区域中,对任给质量都存在具有该质量的径向对称初值使得解有限时刻爆破,并证明了这样的初值在某个合适的拓扑意义下是稠密的。

\NsfcSection{32}{主要研究内容和目标}{}

本课题拟研究一类表型异性生物趋化模型解的爆破问题。

研究内容一:\textbf{(具间接信号产生的抛物-抛物-椭圆趋化模型)}
具间接信号产生的完全抛物趋化模型在三维及以下或四维次临界质量条件下($\int_\Omega u_0< 64\pi^2$)径向对称解都是整体存在的,在超临界质量条件下($\int_\Omega u_0> 64\pi^2$),存在径向对称初值使得解在有限或无穷时刻爆破,但是否存在有限时刻爆破解依然是公开问题。申请人与合作者考虑了以下Nagai型简化问题
\begin{align}
    \begin{cases}
      \label{sys: ks isp ppe}
        u_t =  \nabla\cdot(\nabla u - u\nabla v),& x\in\Omega, t>0,\\
        v_t =  \Delta v - v + w,& x\in\Omega,	 t>0,\\
        0   = \Delta w - w + u, & x\in\Omega, t > 0, 
    \end{cases}
\end{align}
并已经证明了在五维及以上的小球区域中,对任给初始质量都存在具有该质量的径向对称初值使得解有限时刻爆破。研究临界四维空间中该模型能否发生有限时刻爆破对理解间接信号机制具有重要意义。

研究目标一:拟讨论在齐次Neummann边界条件下该模型在四维小球区域中径向对称解的有限时刻爆破问题,并证明对任给超临界质量,都存在具有该质量的径向对称初值使得解有限时刻爆破,并给出刻画有限时刻爆破的初始能量准则,即低能量初值迫使有限时刻爆破发生。

拟解决的关键科学问题一:
如何在间接信号抑制效应下建立四维临界情形时能量泛函与其耗散率之间的次线性泛函不等式;如何在超临界质量条件下构造一族同质量的正初值使之能量没有下界。

拟采用的研究方案一:基于能量耗散泛函,建立能量与其耗散率的次线性泛函不等式并构造低能量初值侦测爆破。利用椭圆正则性理论和热半群理论得到解关于时间一致的Sobolev正则性估计,结合径向对称Sobolev空间在加权函数空间的嵌入关系得到解及其梯度的逐点估计;利用Adams不等式分内外球区域建立能量泛函与其耗散率之间的次线性泛函不等式;在超临界质量条件下,基于Cauchy问题稳态特解构造一族同质量正初始函数,使之能量没有下界。

研究内容二:\textbf{(具间接信号产生的完全抛物趋化模型)}申请人与合作者已经证明了具间接信号产生的完全抛物趋化模型
\begin{align}
    \begin{cases}
      \label{sys: ks isp ppp}
        u_t =  \nabla\cdot(\nabla u - u\nabla v),& x\in\Omega, t>0,\\
        v_t =  \Delta v - v + w,& x\in\Omega,	 t>0,\\
        w_t  = \Delta w - w + u, & x\in\Omega, t > 0, 
    \end{cases}
\end{align}
在五维及以上的小球区域中,对任给初值质量,都存在具有该质量的径向对称初值使得解有限或无穷时刻爆破。但该模型是否存在有限时刻爆破解依然是公开问题。虽然五维及以上是超临界维数,但因为完全抛物问题的时空强耦合性,侦测解的有限时刻爆破依然具有相当的难度。研究具间接信号产生的完全抛物趋化模型解在高维的有限时刻爆破问题可以加深对间接信号的时间松弛机制的理解。

研究目标二:讨论在齐次Neumann边界条件下该模型在五维及以上小球区域中径向对称解的有限时刻爆破问题,并证明对任意质量,都存在具有该质量的径向对称初值使得解有限时刻爆破,并给出刻画有限时刻爆破的初始能量准则,并说明迫使有限时刻爆破的初值在合适的拓扑意义下是稠密的。

拟解决的关键科学问题二:能量耗散泛函中不显含$w_t$项,建立能量与耗散率关系时如何克服$w_t$项带来的时间非局部性困难。

拟采用的研究方案二:基于以往的研究经验,当第三个方程是椭圆问题时,这个问题可以看作空间非局部信号产生模型,能够通过分布积分技巧将间接信号导致的空间非局部问题转化为信号的相关二阶估计。我们仍然用能量方法侦测有限时刻爆破,但无法期望建立时间无关的能量与耗散率之间的泛函不等式。我们需要对$w_t$关于时间分部积分,并因而建立能量与其耗散率时间积分形式的积分不等式。

研究内容三:\textbf{(具间接信号产生的抛物-抛物-ODE趋化方程)}以下具间接信号产生的趋化模型
\begin{align}
    \begin{cases}
      \label{sys: ks isp ppodef}
        u_t =  \nabla\cdot(\nabla u - u\nabla v) + f(u,w),& x\in\Omega, t>0,\\
        v_t =  \Delta v - v + w,& x\in\Omega,	 t>0,\\
        w_t  = - w + u, & x\in\Omega, t > 0, 
    \end{cases}
\end{align}
是为了描述了山松甲壳虫的集群攻击行为,其中$f(u,w)=-u+w$或$f(u,w)=0$表示是否考虑甲壳虫之间的转化机制。现在已经清楚该间接信号产生机制具有时间松弛效应,解在二维总是整体存在的,但依然有无穷时刻爆破的临界质量现象。虽然该模型在高维空间中是超临界的,但是间接信号的时间松弛带来的爆破抑制效果以及ODE方程的低正则性和时间非局部性对侦测爆破带来了巨大的困难,所以高维空间中该模型能否发生有限时刻爆破现象一直是公开问题。研究该模型三维及以上解的有限时刻爆破现象有助于理解山松甲壳虫的聚集行为。

研究目标三:讨论在齐次Neumann边界条件下该模型在三维及以上小球区域中径向对称解的有限时刻爆破问题,证明对任给初始总量,无论是否考虑甲壳虫类型转化机制,都存在具有该质量的径向对称初值使得解有限时刻爆破。

拟解决的关键科学问题三:筑巢甲壳虫的时间演化行为受ODE方程支配,要克服ODE方程的低正则性和时间非局部性带来的困难;间接信号产生机制的爆破抑制效应还体现在低能量初值的构造上,该模型的能量泛函高度耦合,在低维(三维和四维)空间中构造低能量正初值是主要的困难。类型转化机制进一步增加了方程的耦合性,而且$u$不再质量守恒,这对侦测有限时刻爆破带来了一定的困难。

拟采用的研究方案三:采用能量方法,建立能量与其耗散率之间的次线性不等式,但由于间接信号的时间松弛机制,也就是信号产生的时间非局部性,需要对含$w_t$项关于时间分部积分,将低正则性的$w_t$转化为较高正则性的$v_t$,因而需要建立能量与其耗散率之间的时间积分型的不等式。间接信号机制的强耦合性使得能量泛函中含有$\|\Delta v-v+w\|_{L^2}$项,造成在三维和四维空间中构造低能量径向对称正初值变得更加困难,这就需要适当的构造信号聚集度$v$并用非负函数$w$平衡$\Delta v$使得这一项可控。当考虑类型转换机制,在构造低能量初值时要进一步精细的联系$u$和$w$确保对任意初始总量都能构造一族低能量初始函数族。

研究内容四:\textbf{(具间接信号产生的完全抛物趋化模型的粘性消失极限问题)},具间接信号产生的完全抛物($\varepsilon=1$)趋化模型在物理空间中不会发生爆破现象,
\begin{align}
    \begin{cases}
      \label{sys: ks isp ppviscosity}
        u_t =  \nabla\cdot(\nabla u - u\nabla v) + f(u,w),& x\in\Omega, t>0,\\
        v_t =  \Delta v - v + w,& x\in\Omega,	 t>0,\\
        w_t  = \varepsilon \Delta w - w + u, & x\in\Omega, t > 0, 
    \end{cases}
\end{align}
其中$f(u,w)=0$或$f(u,w)=-u+w$。但是当生产信号的种群类型$w$没有运动能力时($\varepsilon=0$),在二维有解区域中仍可能发生无穷时刻爆破现象,并且我们拟证明三维小球区域中仍会发生有限时刻爆破现象。进一步,一个自然的问题就是考虑生产信号的种群类型$w$的运动能力($\varepsilon\in(0,1)$)对物理空间中种群聚集的影响,以及研究三维及以下有界区域中具间接信号产生的完全抛物趋化模型的三维粘性消失极限问题。

研究目标四:探索生产信号的种群表型$w$的弱运动能力($0<\varepsilon\ll 1$)对高密度聚集的自发涌现现象的调控作用,证明存在初值和时刻$T>0$使得函数族$\{u_\varepsilon\}_{\varepsilon\in(0,1)}$在$L^\infty(\Omega\times(0,T))$拓扑意义下关于$\varepsilon$并非一致有界的。研究三维及以下有界区域中具间接信号产生的完全抛物趋化模型的三维粘性消失极限问题,证明存在时间$T\in(0,\infty]$使得整体古典解族$\{(u_\varepsilon,v_\varepsilon,w_\varepsilon)\}_{\varepsilon\in(0,1)}$关于$\varepsilon\to0$在$\Omega\times(0,T)$上在弱的意义下收敛到具间接信号产生的抛物-抛物-ODE趋化模型的唯一弱解,并给出该极限关于$\varepsilon$的收敛速率的估计。

拟解决的关键科学问题四:如何建立与$\varepsilon$无关的在弱解意义下的古典解族相关时空正则性估计。如何通过极限问题二维及三维的爆破结果侦测小扩散问题的高密度聚集的自发涌现现象。

拟采用的研究方案四:对古典解族做$L^p$估计来推导出在弱解意义下的相关时空正则性关于$\varepsilon$一致的定量描述。运用压缩映像原理等不动点方法证明存在一个与$\varepsilon$无关的时刻$T_0$使得上述时空估计在$\Omega\times T_0$上一致成立。应用Arzela-Ascoli 定理证明粘性消失极限的弱解收敛性。由抛物正则性理论可得该收敛性也在强解以及古典解的意义下成立。运用热半群理论估计收敛速率。结合间接信号的时间松弛机制证明在平面区域中$T_0$可以延拓到无穷时刻。结合本项目前面的预期研究结果:存在粘性消失极限解在三维有限时刻爆破,采用反证法可得存在初值使得$T_0$在三维小球区域中不能延拓到无穷,而且当$0<\varepsilon\ll1$时,会发生高密度聚集的自发涌现现象。

\NsfcSection{33}{研究方案和技术路线}{}


\NsfcSection{34}{特色与创新之处}{}

本项目的主要特色和创新之处主要体现在选题、研究内容和方法等三个方面:

(1)选题方面:本项目研究内容取材自生物数学领域,具有鲜明的生物学背景,主要聚焦于生物和数学等领域学者广泛关注的抛物型生物趋化模型,也是非线性偏微分方程领域的研究热点之一。选取的具间接信号产生的生物趋化模型都刻画了表型异性的生物种群受趋化作用做出定向运动的自组织行为。种群表型异性概念新颖,是近年来生物和数学建模的热点话题,有其合理性和科学依据:在描述更复杂和真实的生物聚集现象时具有显著的优势,能够解释一些经典趋化模型无法解释的生物聚集现象。同时,所研究的模型都是强耦合的非线性交叉扩散方程组,研究其解爆破的性质具有一定的挑战性。

(2)研究内容方面:本项目研究的课题聚焦于具间接信号产生趋化模型的爆破机制,着重考察信号生产表型种群能否运动($\varepsilon = 1$或$\varepsilon=0$)对爆破机制的影响,还关注两类模型解的极限关系以及弱运动能力($0<\varepsilon\ll 1$)诱发的高密度聚集的自发涌现现象。由项目申请人对这两类模型简化问题的前期研究基础出发,研究内容层层递进,兼具广度和深度,更有助于深入理解趋化运动与趋化因子生产表型异性对爆破机制与聚集现象的影响。研究的内容更加贴近现实复杂的生物聚集现象,是典型的需求导向型,研究的结果能够为人们进一步理解表型异性在相关的生物与医学过程中的作用,并为病虫害防治、癌症的治疗等现实应用提供理论依据。

(3)研究方法方面:本项目研究的生物趋化方程组具有强非线性和高耦合性,没法通过变量替换转化为标量抛物问题从而运用比较方法,也没法运用一些经典的适用于抛物椭圆型生物趋化模型的加权函数法。这导致理论分析难度大大增加,进而要求在研究过程中,充分利用模型自身的特点,如质量守恒、能量耗散结构及结构形式的对称性等,发展新的方法与分析技巧,进而促进非线性趋化模型理论与工具的发展。

\NsfcSection{35}{研究可行性}{(可行性分析)}

(1)研究方案上。本项目拟采用的能量方法是处理完全抛物趋化模型解爆破的经典方法,即以Lyapunov能量耗散泛函作为基本理论工具,以建立能量与耗散率之间的泛函不等式作为基本指导思想。本项目结合具间接信号产生的生物趋化模型的结构特点,灵活应用热算子半群、Sobolev嵌入不等式及椭圆和抛物正则性理论等技术,来解决研究内容中所提到的问题,做到理论扎实、方法正确、论证严谨、结论清楚。

(2)研究基础上。本项目是申请人博士学位论文基础上的进一步深入与扩展。项目申请人在攻读博士学位期间就以Keller-Segel型生物趋化模型作为研究对象,以解的爆破性质作为主要研究内容,阅读了大量的相关文献,掌握了相关的研究方法,并取得了一些有意义的结果,相关成果已经发表在\textbf{Mathematical Models and Methods in Applied Sciences、Nonlinear Analysis、Discrete and Continuous Dynamical Systems. Series S}等国际知名数学期刊上。申请人具有一定的科研能力和探索精神,同时掌握了本项目所需要的基本理论和研究方法,搜集了与本项目研究主题相关的最新文献,具备开展本项目前期工作基础。

(3)研究资源上。本项目所依托单位

\NsfcSection{36}{研究基础和条件保障}{}

\NsfcSection{39}{项目的立项依据}{}

趋化指生物(包括细菌、细胞及其他单细胞或多细胞生命体)对外界环境中的化学物质刺激所产生的趋向性反应。趋化性是生物的本能反应之一,在自然界中广泛存在,比如细菌向有较高养分浓度的区域游走,或远离有害物质的地方。趋化性不仅对生物觅食、躲避天敌、规避有害物质、求偶等有重要作用,而且在病虫害防治、生物除污、生物医疗、石油开采及病虫害防治等生产生活领域的研究中具有重要意义。

通过化学趋向运动进行聚集的能力形成了自组织系统的范式,这一机制在细胞和动物系统中均有实例。1970年,Keller 和 Segel 为了研究微生物的聚集现象,提出了描述生物趋化过程的抛物型偏微分方程组(经典Keller-Segel趋化模型)
\begin{equation}
    \begin{cases}
        u_t = \nabla \cdot(\nabla u - u\nabla v), \\ 
        v_t = \Delta v - v + u,
    \end{cases}
\end{equation} 
其中$u$表示微生物的种群密度,$v$表示化学信号的浓度。该模型假设一个表型均一的微生物种群自身会分泌趋化因子并对此有趋向吸引反应。该机制在建模研究中仍然极为流行,如基于细胞本身占有一定体积而提出的密度依赖趋化模型(也称为具体积填充效应的趋化模型);基于细胞的有限扩散速度而提出了通量限制趋化模型;还有描述流体对微生物运动影响而提出的趋化-流体模型等。这些模型的提出以及后续的持续深入研究大大丰富了趋化模型的研究内容并推动了相关数学理论和方法的发展。然而,种群表型均质性的典型假设通常与自然系统的异质性相矛盾,在自然系统中,种群可能由不同的表型组成,这些表型根据它们的趋化能力、吸引因子分泌等因素发生变化。

为了理解种群表型多样性如何影响种群自聚集,本课题拟研究一类具间接信号产生的生物趋化模型解的定性性质,着重研究聚集现象与爆破性质。该模型是经典Keller-Segel趋化模型的简单扩展,其中种群被划分为两种不同的表型:一种执行趋化性运动,另一种则生产趋化因子。

%现在已经广为人知,该模型展现了扩散与(趋化诱导)聚集的竞争机制,具有丰富的动力学行为,特别会发生爆破现象——聚集的极端形式——这吸引了大批数学学者研究趋化模型解的定性理论。爆破的发生反映了该模型很好地描述了微生物的聚集现象,同样吸引了大批生物和数学等学者发展生物趋化现象的数学建模方法。爆破的发生反映了该模型很好地描述了微生物的聚集现象,同样吸引了大批生物和数学等学者发展生物趋化现象的数学建模方法。为了更真实地模拟生物趋化现象,一些改进模型被相继提出。

%为了更真实地模拟生物趋化现象,一些改进模型被相继提出。例如基于细胞本身占有一定体积而提出的密度依赖趋化模型,也称为具体积填充效应的趋化模型;基于细胞的有限扩散速度而提出了通量限制趋化模型;还有描述流体对微生物运动影响而提出的趋化-流体模型等。这些模型的提出以及后续的持续深入研究大大丰富了趋化模型的研究内容并推动了相关数学理论和方法的发展。这些模型的一个基本假设是表型同质化的种群分泌自身的化学引诱剂,然而,种群表型均质性的典型假设通常与自然系统的异质性相矛盾,在自然系统中,种群可能由不同的表型组成,这些表型根据它们的趋化能力、吸引物分泌等因素发生变化。。例如生物分雌雄个体,只有雌性个体会释放费洛蒙,且只有雄性个体会展示对费洛蒙的趋向运动。例如生物种群内分成体和幼体、分工不同等都会导致个体间有不同的运动行为模式。

%生物趋化模型扩散与(趋化诱导)聚集的竞争机制使之具有丰富的动力学行为,特别是爆破现象——聚集的极端形式——吸引了大批数学学者研究趋化模型解的定性理论。

%通过化学敏感性运动进行聚集的能力形成了自组织的范式,且这一机制在细胞和动物系统中均有实例。一个基本机制假设一个表型均质的种群分泌自己的吸引物,其中凯勒(Keller)和塞格尔(Segel)提出的著名系统在五十多年前被引入,并在建模研究中仍然极为流行。然而,种群表型均质性的典型假设通常与自然系统的异质性相矛盾,在自然系统中,种群可能由不同的表型组成,这些表型根据它们的趋化能力、吸引物分泌等因素发生变化。为了启动对这种多样性如何影响自聚集的理解,我们提出了一个简单的扩展,基于经典的凯勒-塞格尔模型,其中种群被划分为两种不同的表型:一种执行趋化性运动,另一种则生产吸引物。通过结合线性稳定性分析和数值模拟,我们证明了在这些表型状态之间的切换改变了种群自聚集的能力。进一步地,我们展示了基于局部环境(如种群密度或趋化因子浓度)的切换导致了多样化的模式,并提供了一种途径,使得种群能够有效地抑制聚集体的大小和密度。我们还在真实的趋化聚集实例以及模型的理论方面(如解的全局存在性和爆炸)讨论了这些结果。


%本课题拟研究一类具间接信号产生的生物趋化模型解的爆破现象。该类模型将生物种群分为两类,一类产生趋化吸引子,不会展现趋化运动;另一类不产生趋化吸引子,但会作出趋化运动;种群内部两类种群之间具有转换机制。

%%%%%%%%%%%%%%%%%%%%%%%%%%%%%%%%%%%%%%%%%%%%%%%%%
\NsfcSection{2}{项目的研究内容、研究目标,以及拟解决的关键科学问题}{
(此部分为重点阐述内容);}

\subsection{研究目标}


重点阐述本项目计划\myEmph{做到什么程度}。
一定要用大同行能够理解的术语来描述。


\subsection{研究内容}


重点阐述本项目计划\myEmph{做什么}。
如\figref{fig:teaser}所示,建议撰写之前仔细画一个主要研究内容的图。
我通常习惯用PowerPoint作图,做好后导出为pdf格式,既可以保持图片不会文件太大,也可以保证放大后非常清晰。
幻灯片直接导出的pdf可能存在空白边,可以用WPS中的“页面-剪裁页面”去掉白边。
作图用的pptx文件我也通常会保存起来,例如这个模版\LaTeX 文件中的“prepare/NSFC-Figs.pptx”。
如果是大的国基金项目,后续可能涉及答辩。
答辩时这个图可以用,保留pptx格式也方便到时候做尺寸和布局的调整。


\begin{figure}[ht]
	\centering
    \begin{overpic}[width=0.8\columnwidth]{framework.pdf}
    \end{overpic}
    \caption{本项目主要研究内容。
    }\label{fig:teaser}
\end{figure}




\subsection{拟解决的关键科学问题}



\NsfcSection{3}{拟采取的研究方案及可行性分析}{
(包括研究方法、技术路线、实验手段、关键技术等说明);}


\subsection{拟采取的技术路线}

如图\figref{fig:pipline}所示,建议在具体写技术路线之前,先厘清这个框架图。
技术路线的框架图通常比研究内容更加饱满,可以更好的展示本项目的研究思路。

\begin{figure}[ht]
	\centering
    \begin{overpic}[width=\columnwidth]{pipeline.pdf}
    \end{overpic}
    \caption{本项目的技术路线。
    }\label{fig:pipline}
\end{figure}


\subsection{可行性分析}

既然国内外相关工作都没能解决你提出的重要问题,为什么你觉得自己有望解决该问题。
论述的时候通常包括:独特的时机、与众不同的方案、雄厚的相关科研积累等。


\NsfcSection{4}{本项目的特色与创新之处;}{}

您的方法有什么独特性,为什么您认为它会成功?

\NsfcSection{5}{年度研究计划及预期研究结果}{
(包括拟组织的重要学术交流活动、国际合作与交流计划等)。}

\subsection{年度研究计划}


\subsection{预期研究成果}

项目各阶段,特别是中期和期末,如何检查该项目计划成功与否?

%%%%%%%%%%%%%%%%%%%%%%%%%%%%%%%%%%%%%%%%%%%%%%%%%
\ContentDes{(二)研究基础与工作条件}


\NsfcSection{1}{研究基础}{
(与本项目相关的研究工作积累和已取得的研究工作成绩);}

以我本人的第一个国基金项目的\myEmph{申请原文为例},通过高度精炼的语言,
在2页左右的篇幅中快速讲清楚:1)大部分申请人没有的特色优势是什么;
2)特色工作和拟研究工作的联系得清晰。

申请人在与本项目相关的研究工作中有着丰富的积累,并取得了国际领先的科研成果, 
发表\myEmph{10余篇CCF A类}国际期刊及会议论文(ACM TOG 4篇,IEEE TPAMI 2篇,
IEEE CVPR 3篇,IEEE ICCV 2篇,IEEE TVCG 2篇),
\myEmph{论文Google Scholar他引1600+次,一作论文单篇最高他引700+次}。
这些研究基础主要包括3个方面:


\subsection{图像的智能理解与新一代交互方面}

\begin{figure}[ht]
    \centering
    \addImg[1]{figures/work.jpg}
    \caption{图像智能理解与交互关键技术。}
    \label{fig:interaction}
\end{figure}

对图像中的物体对象进行智能分析,从而实现基于语音、简单笔画等自然直观操作方式的图像编辑,
是计算机图形学中的重要课题。
如\figref{fig:interaction}所示,在该领域,
申请人提出了基于图像场景物体的编辑技术
(ACM TOG 2010 )%\cite{ChengZMHH10},申请人为一作),
基于语音控制的图像分析技术 
(ACM TOG 2014 )%\cite{cheng2014imagespirit},申请人为一作),
基于草图和网络图像的图像合成技术
(ACM TOG 2009 )%\cite{cheng2014imagespirit},申请人为二作)。
其中,图像合成技术于2010年被法国政府参与组织的国际互联网行业论坛评选为
“全球互联网数字媒体领域十大创新性发明之一”并在法国参议院颁奖,
成果介绍视频浏览百万次以上,并被英国《每日邮报》、
德国《明镜周刊》等多家著名国际媒体撰文报道。
英国《每日邮报》对该工作的评价为“An image manipulation tool built by a group 
of Chinese students has taken the internet by storm”。
该工作被科技部 %\cite{sketch2PhotoMost} 
\myEmph{作为973计划成功案例}予以报道(信息领域仅列出这一个案例)。
申请人在该领域的CCF A类国际期刊ACM TOG(4篇)和IEEE TVCG(2篇)上发表多篇论文。

\subsection{显著性物体检测方面}

\begin{figure}[ht]
    \centering
    \addImg[1]{figures/sal.jpg}
    \caption{视觉显著性物体检测与分割算法及其应用。}
    \label{fig:salobj}
\end{figure}

申请人提出了一种基于全局对比度分析的图像视觉显著性区域检测算法。
通过对图像区域间的全局对比度和空间相关性进行建模,
该算法能够快速有效地检测并分割图像中的视觉显著性区域。
在国际上现有最大的公开测试集上,该方法的检测结果优于已有方法,
显著性区域分割结果的准确性从之前最好结果的正确率75\%、
召回率83\% 提升到了正确率90\%、召回率90\% 。
该成果自2011年在IEEE CVPR上发表之后(申请人为一作),源代码下载量2000余次,
论文\myEmph{Google Scholar他引700+次}(其期刊版本被IEEE TPAMI 2015录用)。
基于该工作的若干扩展性工作在IEEE ICCV 2013(申请人为一作),
和SCI期刊The Visual Computer 2014(申请人为一作)上发表。


\subsection{基于视觉注意机制的候选物体生成方面}

\begin{figure}[ht]
    \centering
    \addImg[.8]{figures/Objectness.jpg}
    \caption{图像中候选物体(object proposal)生成的快速机制。}
    \label{fig:objectness}
\end{figure}

物体检测是计算机视觉领域最重要的核心问题。
传统物体检测技术大多基于滑动窗口机制,非常耗时。
近年来,通过对图像进行预分析,从而提取相对少量的图像窗口作为候选物体,
并对这些候选物体进行分类从而达到检测目的的机制逐渐兴起。
虽然这一机制可以极大地加速后续物体检测过程,
并且通过使用强分类器提高检测正确率,但是现有方法比较耗时。
申请人根据不同物体类别图像都具有封闭轮廓这一特性,
提出了一种简单有效的方法解决这一问题。
相对与该领域内已有代表性方法,新方法的计算效率提升了1000倍以上,
并能得到更高的检测率。
\figref{fig:objectness}所示工作被IEEE CVPR 2014录用为Oral 论文
(录取率5\%,申请人为一作)。
相关代码公开后一个月内下载量突破2000次。
论文发表半年左右时间,已经被多个国际著名学者在论文或者大会报告中引用,
包括:IEEE TPAMI主编 Forsyth教授、
加州大学伯克利分校的 Malik教授(IEEE Fellow和ACM Fellow)、
牛津大学 Vedaldi副教授(IEEE TPAMI Associate Editor)、
瑞士联邦工学院的Gool教授等。
申请人也因为这方面的工作被英国皇家学会院士Andrew Zisserman教授、
剑桥大学Roberto Cipolla教授、新加坡国立大学颜水城教授等十余位著名学者邀请,
分别去他们的研究组做报告。


\subsection{研究工作获奖}


\begin{figure}[ht]
    \centering
    \addImg[.8]{figures/awards.jpg}
    \caption{图像中候选物体(object proposal)生成的快速机制。}
    \label{fig:award}
\end{figure}

申请人在图像分析与编辑方面的初步工作获得了多项重要奖励,
包括:Google PhD Fellowship (2010年亚洲地区共2人获奖),
北京市优秀博士论文(同年度清华大学56个系共有7人获得该奖项),
英国皇家学会Newton International Fellowship提名,教育部自然科学奖一等奖,
IBM PhD fellowship,北京市优秀博士毕业生,教育部博士研究生学术新人奖等奖项。
这些初步研究中取得的成果为本项目取得进一步的创新性成果打下了坚实的基础。


\NsfcSection{2}{工作条件}{
(包括已具备的实验条件,尚缺少的实验条件和拟解决的途径,
包括利用国家实验室、
国家重点实验室和部门重点实验室等研究基地的计划与落实情况);}


\myPara{申请平台概况}
南开大学计算机学科在计算机视觉与计算机图形学领域具有非常扎实的研究基础。
作为教育部直属重点大学,南开大学是国内学科门类最齐全的综合性、研究型大学之一。
在计算机视觉与图形学方向上,
南开大学计算机学院拥有计算机与控制工程国家级虚拟仿真实验教学中心、
可信行为智能算法与系统教育部工程研究中心、和天津市视觉计算与智能感知重点实验室
等一系列优良研究平台。
计算机视觉与图形学团队现拥有一批先进的高性能GPU服务器集群(包含Tesla A40, Tesla V100, RTX 3090等高端GPU共计900多块)。
% 
为本项目的研发提供强有力的计算环境。
南开大学为计算机视觉与图形学团队提供了1200平米的科研用房。
这些良好的配套支持将为项目的顺利开展提供优良的工作条件。


\myPara{科研团队介绍}

申请人所在的南开大学计算机视觉与图形学团队
由国家杰出青年基金获得者程明明教授带领,
包含国家级“四青”人才5人。
%
近五年,团队承担国家自然科学基金重点项目、国家重点研发计划课题、
国防科技创新重点项目等重点重大项目10余项;
在相关领域的 SCI 一区/CCF A 类顶级国际期刊和会议上发表学术论文100余篇,
其中 TPAMI论文 40余篇,ESI高被引论文30余篇;
获得教育部自然科学一等奖、中国图象图形学学会自然科学一等奖、
吴文俊人工智能自然科学二等奖等多项奖励。


\myPara{国内外合作与学术交流情况}

南开大学计算机视觉与图形学团队有着丰富的国内外学术合作与交流基础。
近年来,团队与计算机视觉最高奖Marr奖得主、英国皇家学会院士、
牛津大学Philip Torr教授团队,
计算机视觉最高奖Marr奖得主、加州大学圣迭戈分校的Zhuowen Tu教授团队,
和清华大学胡事民教授团队合作开展了多项具有影响力的学术研究,
并共同发表相关学术论文。
团队成员受邀担任IEEE TPAMI, IEEE TIP, IEEE CVPR, ICCV等本领域
多个顶级期刊和会议的编委或领域主席。
团队成员作为主要组织者承办了VALSE 2022 (3000余人现场参会),2023 (5000余人现场参会),
和PRCV 2020,2024 (1000多人现场参会)
等多个国内大型学术会议。
%
与这些国内外相关研究团队开展密切合作与交流的经验也将为项目的成功
进行起到重要的促进作用。

申请人所申报的项目将依托于南开大学计算机视觉与图形学团队开展。
现有优越的实验条件可以充分满足项目研发需求。
项目团队将继续加强国际交流与合作,
以创新性的高水平学术成果为研究目标。


\NsfcSection{3}{正在承担的与本项目相关的科研项目情况}{
(申请人正在承担的与本项目相关的科研项目情况,包括国家自然科学基金
的项目和国家其他科技计划项目,要注明项目的资助机构、项目类别、
批准号、项目名称、获资助金额、起止年月、与本项目的关系及负责
的内容等);}

正在承担的项目

\NsfcSection{4}{完成国家自然科学基金项目情况}{
(对申请人负责的前一个已资助期满的科学基金项目(项目名称及批准号)完成情况、后续研究进
展及与本申请项目的关系加以详细说明。另附该项目的研究工作总结
摘要(限500字)和相关成果详细目录)。}

已完成的国家自然科学基金

%%%%%%%%%%%%%%%%%%%%%%%%%%%%%%%%%%%%%%%%%%%%%%%%%
\ContentDes{(三) 其他需要说明的问题}



\NsfcSection{1}{}{
申请人同年申请不同类型的国家自然科学基金项目情况(列明
同年申请的其他项目的项目类型、项目名称信息,并说明与本项目之
间的区别与联系;已收到自然科学基金委不予受理或不予资助决定的,
无需列出)。}


\NsfcSection{2}{}{
具有高级专业技术职务(职称)的申请人是否
存在同年申请或者参与申请国家自然科学基金项目的单位不一致的情
况;如存在上述情况,列明所涉及人员的姓名,申请或参与申请的其
他项目的项目类型、项目名称、单位名称、上述人员在该项目中是申
请人还是参与者,并说明单位不一致原因。}



\NsfcSection{3}{}{
具有高级专业技术职务(职称)的申请人是否
存在与正在承担的国家自然科学基金项目的单位不一致的情况;如存
在上述情况,列明所涉及人员的姓名,正在承担项目的批准号、项目
类型、项目名称、单位名称、起止年月,并说明单位不一致原因。}


\NsfcSection{4}{}{同年以不同专业技术职务(职称)申请或参与申请科学基金项
目的情况(应详细说明原因)。}

\NsfcSection{5}{}{其他。}

无


\end{document}


% 雷震,复旦大学数学科学学院教授,国家级高层次人才计划入选者。他因在流体方程组解
% 的定性理论方面的系列工作获得了2020年度国家自然科学二等奖、科学探索奖等多项荣誉。
% 雷震提出了“强零条件”的概念,发现了不可压流体方程组的非线性内蕴强退化结构,最终
% 取得突破性进展,独自解决了二维弹性力学方程组解的整体适定性这一长期公开问题。这
% 一结构和所发现的非线性恒等式成为该领域的重要研究基础,致使他与合作者随后建立了粘
% 弹方程组解的整体粘性消失理论和弹性力学方程组自由边值问题解的局部适定性理论。在不
% 可压Navier-Stokes方程组方面,雷震与合作者得到了BMO-1空间中轴对称解的整体适定
% 性和古代解的Liouville性质,建立了奇点集合的维数理论和奇点可去性的几何化判别法。
